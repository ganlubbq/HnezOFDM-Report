GNURadio \cite{gnuradioweb} is a set of free software \cite{fsffreeweb}
tools that are useful for building and simulating radio systems.
This chapter presents the anatomy of a GNURadio module.

\begin{subchapter}{GNURadio Companion}
  The first point of contact for most new GNURadio users
  is usually the \gls{grc}, a program that lets
  users design signal flowgraphs in a graphical way.
  \Gls{grc}'s main interface is shown in figure
  \ref{img:annotated_gnuradio_companion}, the flowgraph
  is an implementation of the \acrlong{schcox} detector block-diagram
  shown in figure \ref{img:sc_detector_blocks} in the previous
  chapter.

  \figurizefile{diagrams/annotated_gnuradio_companion.tex}
               {img:annotated_gnuradio_companion}
               {The \acrlong{grc}}
               {1}{ht}

  When a flowgraph containing graphical instrumentation like
  the \texttt{QT GUI Time Sink} in figure \ref{img:annotated_gnuradio_companion}
  is executed, a window will open containing the output of
  the instruments.

  Figure \ref{img:gnuradio_gui} shows such a window, containing
  a \texttt{QT GUI Time Sink}, a GNURadio oscilloscope.

  \figurizegraphic{images/gnuradio_gui.png}
                  {img:gnuradio_gui}
                  {The graphical output of a running flowgraph}
                  {0.8}{H}
\end{subchapter}
