\begin{subchapter}{Single carrier transmision}
  A classic digital transmission standard will usually map
  digital bits to complex symbols and will modulate these
  symbols at a fixed rate onto a single carrier frequency. \\

  An example of six \acrshort{qam4} symbols that are transmitted
  in equal time intervals can be seen in figure \ref{img:qam4_symbol_time}.
  The Symbols are shown in the complex baseband domain and the
  hypothetical filter is constructed such, that the signal
  phase is interpolated linearily between symbols.

  \figurizefile{diagrams/qam4_symbol_time.tex}
               {img:qam4_symbol_time}
               {\acrshort{qam4} modulated symbols over time in the complex domain}
               {1}{H}

  Clock recovery and synchronizing to a continious stream of
  symbols are, at least in theory, straightforward.

  When only one of the complex components (real- or imaginary part)
  is observed, as shown in figure \ref{img:eye_diagram}, one can
  immidiately guess the optimal sampling time, indicated by the
  vertical line in the middle of the plot.
  The optimal sampling time can be found by observing the
  zero-crossings of the signal which have to happen half a
  sybol period after the sampling.

  In low noise situations, as shown in \ref{img:eye_diagram},
  the modulated digital signal can be perfectly reconstructed.

  \figurizegraphic{diagrams/eye_diagram.pdf}
                  {img:eye_diagram}
                  {Eye diagram}
                  {0.7}{H}

  A phenomenom that is quite common in terristrial communications
  is multipath propagation.
  What this means is that a signal that was transmitted once reaches the
  receiver multiple times delayed by different amounts of time.

  Figure \ref{img:multipath} shows an example of a scenario where
  multipath propagation is happening.
  The Transmitter does not have a direct line of sight to the receiver,
  so signals that reach the receiver did so by being reflected
  on objects.

  In the depicted scenario there are two paths by which signals from
  the transmitter can reach the receiver.
  The path in which the signal is reflected by the object on
  the top is longer than the other path, this means the signal will
  be delayed longer and damped more.

  \figurizefile{diagrams/multipath.tex}
               {img:multipath}
               {A Multipath scenario without a direct line of sight}
               {0.5}{H}

  Figure \ref{img:eye_diagram_multipath} shows the effect of the
  two signals superimposing on one another.
  The signal that follows the longer transmission path
  is delayed by $0.9$ symbol durations and has a magnitude
  of $\SI{10}{\percent}$, $\SI{30}{\percent}$, $\SI{50}{\percent}$ and
  $\SI{70}{\percent}$ of the original signal amplitude.

  % TODO: Phasendrehung und alles ist gar nicht beachtet
  \figurizefile{diagrams/eye_diagram_multipath.tex}
               {img:eye_diagram_multipath}
               {Eye diagrams}
               {0.9}{H}

  With increasing amplitude of the reflected signal the decoding
  and clock recovery become more and more difficult.
  To reduce the effects of multipath propagation one can
  increase the symbol durations, but this will in turn
  reduce the data rate, limiting the data throughput.

\end{subchapter}

\begin{subchapter}{Multicarrier transmision}

\end{subchapter}
