\begin{subchapter}{Single carrier transmision}
  A classic digital transmission standard will usually map
  digital bits to complex symbols and will modulate these
  symbols at a fixed rate onto a single carrier frequency. \\

  An example of six \acrshort{qam4} symbols that are transmitted
  in equal time intervals can be seen in figure \ref{img:qam4_symbol_time}.
  The Symbols are shown in the complex baseband domain and the
  hypothetical filter is constructed such, that the signal
  phase is interpolated linearily between symbols.

  \figurizefile{diagrams/qam4_symbol_time.tex}
               {img:qam4_symbol_time}
               {\acrshort{qam4} modulated symbols over time in the complex domain}
               {1}{H}

  Clock recovery and synchronizing to a continious stream of
  symbols are, at least in theory, straightforward.

  When only one of the complex components (real- or imaginary part)
  is observed, as shown in figure \ref{img:eye_diagram}, one can
  immidiately guess the optimal sampling time, indicated by the
  vertical line in the middle of the plot.
  The optimal sampling time can be found by observing the
  zero-crossings of the signal which have to happen half a
  sybol period after the sampling.

  In low noise situations, as shown in \ref{img:eye_diagram},
  the modulated digital signal can be perfectly reconstructed.

  \figurizegraphic{diagrams/eye_diagram.pdf}
                  {img:eye_diagram}
                  {Eye diagram}
                  {0.7}{H}



  \figurizefile{diagrams/multipath.tex}
               {img:multipath}
               {A Multipath scenario without a direct line of sight}
               {0.5}{H}

  \figurizefile{diagrams/eye_diagram_multipath.tex}
               {img:eye_diagram_multipath}
               {Eye diagrams}
               {0.9}{H}


\end{subchapter}

\begin{subchapter}{Multicarrier transmision}

\end{subchapter}
